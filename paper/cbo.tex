%%%%%%%%%%%%%%%%%%%%%%%%%%%%%%%%%%%%%%%%%%%%%%%%%%%%%%%%%%%%%%%%%%%%%%%%%%%%%%%
\subsubsection{Cardinality and cost estimation}
\label{sec:cardinality}
We design a cardinality and cost estimator to estimate the query's output size and execution cost. Just as traditional work, the system analyzes relational operators and uses some pre-computed statistics data of the input tables.

In our work, computing the query's output size mainly relies on the statistics data, which are computed in 2 levels: relation and column. In relation level, the system obtains the number of records and average record size. In more detail level - column, the system collects the min, max, the cardinality and builds an equi-width histogram for each column. The output size of each operator is estimated under the uniform distribution assumption. Although simple, we just want a good estimation to avoid obviously bad plans. More complex histogram techniques could be used to improve the estimation accuracy.

The executing cost comprises of CPU, disk I/O and network costs. Those costs are the results of the multiplication between the pre-defined constants and the estimated number of records.

%%%%%%%%%%%%%%%%%%%%%%%%%%%%%%%%%%%%%%%%%%%%%%%%%%%%%%%%%%%%%%%%%%%%%%%%%%%%%%%

%%%%%%%%%%%%%%%%%%%%%%%%%%%%%%%%%%%%%%%%%%%%%%%%%%%%%%%%%%%%%%%%%%%%%%%%%%%%%%%
\subsubsection{Cache plans selection}
\label{sec:cbo-o}
Our problem is to select the best combination of CEs to form the \emph{cache plans}. For each \emph{CE}, we compute its (profit, weight). Profit of a CE is defined in Section \ref{sec:sharing-vs-notsharing} and weight is the estimated memory footprint of that CE when it is materialized to RAM. Let $W$ be the total cache capacity of the cluster. Then our optimization problem is to maximize the $\sum profit$ under a limited memory capacity: $\sum weight \leq W$. Then our problem is equivalent to the Multiple-choice Knapsack problem (MCKP) and could be solved by many approaches \cite{•}.

Consider $m$ mutually disjoint classes $N_1, N_2, .. N_m$ of items to be packed into a knapsack of capacity $c$. Each item $j \in N_i$ has a profit $p_{ij}$ and a weight $w_{ij}$. The MCKP is to choose at most one item from each class such that the profit sum is maximized without exceeding the capacity $c$ in the corresponding weight sum. The problem is formulated as: 
\begin{align*}
maximize \sum_{i=1}^{k}\sum_{j \in N_i} p_{ij}x_{ij}\\
subject\ to \sum_{i=1}^{k}\sum_{j \in N_i} w_{ij}x_{ij} \leq c\\
\sum_{j \in N_i}x_{ij} \leq 1, i = 1,2,..m\\
x_{ij} \in \{0, 1\}, i = 1,2,..m, j \in N_{i}
\end{align*}

MCKP is a NP-Hard problem. We apply a heuristic algorithm for our system for efficiency \cite{•}.
%%%%%%%%%%%%%%%%%%%%%%%%%%%%%%%%%%%%%%%%%%%%%%%%%%%%%%%%%%%%%%%%%%%%%%%%%%%%%%%
