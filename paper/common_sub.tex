The goal of this phase is to quickly identify all potential sharing opportunities in a query and among different queries. The input of this phase is a set of logical plans that are already parsed, analyzed and optimized individually. A logical plan is the internal form representing a query. It is a tree with nodes are logical operators supported by the execution engine. Each operator has its own attribute(s) (filter predicate, project columns, joining condition, etc.).

Our optimization process starts with the optimized logical plans (right before being transformed into physical plans) as the input. Note that we simplified the problem by only focusing on the locally optimal query plans that can derive globally better ones. The first reason is that all queries are already optimized by the same techniques (for example the rule-based optimizer) and usually, the selection and projection operators are pushed to the relations as close as possible. We want to quickly obtain and identify the potential sharing opportunities without paying too much the cost of taking into consideration each single generated logical plan for each query as in \cite{zhou2007efficient}. Keeping that in mind, it is the target of this step to quickly identify and produce reasonable good candidates.

We begin with finding the similar subexpressions in the trees. As being discussed in Section \ref{sec:problem}, we need to strike a balance between (low) memory utilization and (high) benefits from caching. Considering a query has only selection and projection operators, then the higher the node is, the more selective (less output data) it will be. In other words, some SEs can be safely eliminated from consideration because sharing (caching) them is always worse than some others. However, we may have multiple options if the tree contains one of the following operators: union, cartesian product and join. They are the operators that could possibly produce many output data, even more than the input. Such operators are classified into the \emph{cache-unfriendly operators} group. The rest are \emph{cache-friendly operators} (Filter, Project, Sort, Limit, etc.). Going back to the example in Section \ref{sec:problem}, because SE2, SE3 and SE4 contains only \emph{cache-friendly operators}, we stop looking at the smaller subexpressions for more sharing possibilities. SE1, however, has one \emph{cache-unfriendly operator} - Join. Then we can either select this candidate (SE1) or the 2 smaller ones (SE2 and SE3). We may not immediately conclude which selection is the best. It depends on multiple criteria (how much we can save in each case, how much data each produces and the memory available, etc.). Potential options are: $\{SE1, SE2, SE3, (SE2, SE3)]\}$.

We use the \emph{operator fingerprints} as a mean to detect the similar subexpressions. Found SEs  represents the potential candidates for building CE in the next phase. We first describe how to compute it and how to produce only the reasonable good SE candidates. The fingerprint $FP(u)$ of an operator $u$ is computed by:
\[FP(u)= h(H(u) | Sort(FP(u.lchild), FP(u.rchild))) (1)\]
where $h$ is a robust hash function (for example SHA256) and 
\[H(u)=
\begin{cases}
 & h(u.label),\ u= \{Filter, Project\}\\ 
 & h(u.label, u.attributes)),\ otherwise
\end{cases} (2)\]

Node $u$ in the operator tree has its label $u.label$ (operator name) and attributes $u.attributes$ (filter predicate, projection columns, joining conditions, etc.). The $Sort$ in (1) ensures the isomorphic property, for example $TableA\ JOIN\ TableB$ and $TableB\ JOIN\ TableA$ are two isomorphic expressions. If $u$ is a binary node, we have $u.lchild$ and $u.rchild$ as the left and right child. If $u$ is a unary node or a leaf node, (1) will become $FP(u)= h(H(u), FP(u.child))$ and $FP(u)= h(H(u))$ respectively. Note that the fingerprint of a node (an operator) is computed after computing fingerprint of its child(s). Two operators (and its descendants) have the same fingerprint then we call them similar subexpressions. They must have the same tree structure and property. Treating the Filter and Project operations differently in (2) allows us to identify SEs having different filtering predicates and projections. In phase 2, they can be transformed into equivalent expressions sharing a common subexpression. For instance $e1 = Filter_{a>10}(x)$ and $e2 = Filter_{b<30}(x)$ are similar subexpressions, then $e1, e2$ can be transformed into $Filter_{a>10}(Filter_{a>10\ \cup \ b < 30}(x))$ and $Filter_{b<30}(Filter_{a>10\ \cup \ b < 30}(x))$ without changing the expression's result.

Now that we have a mean to compare two operator trees, the following algorithm search for matchings while avoid producing worse candidates.

\begin{algorithm}
	\caption{Build hash tree}\label{sec:buildht_alg}
	Input: a logical plans (tree)\\
	Output: HashMap[node, fingerprint]
	\begin{algorithmic}[1]
		\Procedure{$BuildHashTree([T])$}{}
		\State $HT:\{(node, fingerprint)\}=$ empty hash tree
		\For{each node $u$ in $T$}
			\State compute fingerprint $opFT$ for $u$ using (1) and (2)
			\State put $(u, opFT)$ into $HT$
		\EndFor
		\State \Return  $HT$
		\EndProcedure
	\end{algorithmic}
\end{algorithm}

\begin{algorithm}
\caption{Identify similar subexpressions}\label{sec:common_sub_alg}
Input: Array of logical plans (trees)\\
Output: List of SEs put together in group
\begin{algorithmic}[1]
\Procedure{$IdentifySEs([T_{1}, T_{2}, ... T_{N}])$}{}
\State $FT:\{(fp, [nodes])\} =$ empty fingerprint table
%\State $common\_sub\_expressions = \{\}$
\For{each tree $T_{i}, i = 1..N$}
	\State$HT(i) = BuildHashTree(T_{i})$
	\State $AllowedMatching = True$	
	\For{each node $u$ in $T_{i}$ follow the DFS traversal}
		\State $opFP = HT(i)(u)$
		\State $IsMatched = FT$ contains $opFP$
		\If {$AllowedMatching$}
				\State Add $(opFP, u)$ entry to $FT$
		\EndIf
		\State $AllowedMatching = True$	
		\If {$IsMatched$ $AND$ the tree rooted at u does not has cache-unfriendly operator}
			\State Stop the search on u's descendants
		\ElsIf {$IsMatched$ $AND$ the tree rooted at u has cache-unfriendly operator $AND$ $u$ is cache-friendly operator}
			\State $AllowedMatching = False$	
		\EndIf		
	\EndFor
	
\EndFor
\State remove any entries (fp, [nodes]) in FT that has length(nodes) = 1
\State \Return  $FT$
\EndProcedure
\end{algorithmic}
\end{algorithm}

By using the expression (1) and (2), the algorithm \ref{sec:buildht_alg} builds a HashTree for each input tree (line 4) of algorithm \ref{sec:common_sub_alg}. Algorithm \ref{sec:buildht_alg} just shows the simplified pseudo code. (Dynamic programming can be applied to save the computations in the for loop). A single fingerprint table (line 2) is used for tracking the matches. Trees with the same fingerprint (same key) will be put in the same group. Following the depth-first-search, each tree and its \emph{operator fingerprint} are added to the fingerprint table (line 10). As being discussed, if we encounter a match on a tree having only cache-friendly operators, then we can safely skip the search on the descendants (line 13 and 14). A group of SEs is detected whenever that group has 2 or more SEs (line 20). Algorithm \ref{sec:common_sub_alg} has O(N*M) complexity where N is the number of trees and M is the average number of nodes per tree.