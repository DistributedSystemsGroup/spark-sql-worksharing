This section describes how the cache primitives can be applied to the context of work-sharing in large-scale distributed computing engines. The whole optimization process containing 4 phases is described in Figure \ref{fig:phases_mqo}. Queries submitted by multiple users are parsed, analyzed and individually optimized normally by the query optimizer. We take the optimized logical plan (the optimized operator tree) of each query as the input to our system. In this paper, we are using the term \emph{logical plan} as the logical representation of a query. Those (optimized) logical plans are then sent to a central server where we will apply our optimization.

\begin{figure}[!htb]
	\centering
 	\includegraphics[scale=0.65]{figures/phases_mqo}
   	\caption{Proposed architecture.} 
   	\label{fig:phases_mqo}
\end{figure}


\textbf{Phase 1: Similar subExpressions(SEs) identification}\\
The goal of this phases is to quickly identify all potential sharing opportunities inside a query and among different queries. Similar subexpressions are subqueries that could benefit from some common computation. We compute the \emph{operator fingerprints} for all logical plans and store them into a fingerprint table. An operator fingerprint is computed by traversing in post order the tree rooted at that operator. Two operators (and its descendants) have the same fingerprint then we call them similar subexpressions. We will discuss this technique in more detail in Section \ref{sec:common_sub}. Found similar subexpressions represents the potential candidates for building covering subexpressions in the next phase.

\textbf{Phase 2: Building Covering SubExpressions}\\
Given the sets of similar subexpressions, the optimizer first tries to eliminate obviously bad candidates for sharing with the help from the cost estimation. Then for each set of subexpressions, we construct a single covering subexpression that is the common computation among those expressions. Some CEs might become \emph{cache plans} after the cost-based optimization in phase 3.

\textbf{Phase 3: Cost-based optimization}\\
This phase achieves the objective of selecting the best combination of CEs which are then become \emph{cache plans}, taking into account the memory constraints and the cost of the caching operation. By using our cost estimation, each CE in previous phase would be assigned a weight and a profit. Finding the best \emph{cache plans} is the most important materialization of our idea to achieve high performance work sharing. 

\textbf{Phase 4: Query rewriting}\\
Last but not least, the input queries will be rewritten such that the cache plans are employed. This step involves some basic query transformations on the original queries.

%%%%%%%%%%%%%%%%%%%%%%%%%%%%%%%%%%%%%%%%%%%%%%%%%%%%%%%%%%%%%%%%%%%%%%%%%%%%%%%
\subsection{Common subexpression identification}
\label{sec:common_sub}
The goal of this phase is to quickly identify all potential sharing opportunities in a query and among different queries. The input of this phase is a set of logical plans that are already parsed, analyzed and optimized individually. A logical plan is the internal form representing a query. It is a tree with nodes are logical operators supported by the execution engine. Each operator has its own attribute(s) (filter predicate, project columns, joining condition, etc.).

Our optimization process starts with the optimized logical plans (right before being transformed into physical plans) as the input. Note that we simplified the problem by only focusing on the locally optimal query plans that can derive globally better ones. The first reason is that all queries are already optimized by the same techniques (for example the rule-based optimizer) and usually, the selection and projection operators are pushed to the relations as close as possible. We want to quickly obtain and identify the potential sharing opportunities without paying too much the cost of taking into consideration each single generated logical plan for each query as in \cite{zhou2007efficient}. Keeping that in mind, it is the target of this step to quickly identify and produce reasonable good candidates.

We begin with finding the similar subexpressions in the trees. As being discussed in Section \ref{sec:problem}, we need to strike a balance between (low) memory utilization and (high) benefits from caching. Considering a query has only selection and projection operators, then the higher the node in that tree is, the more selective (less output data) it will be. In other words, some SEs can be safely eliminated from consideration because sharing (caching) them is always worse than some others. However, we may have multiple options if the tree contains one of the following operators: union, cartesian product and join. They are the operators that could possibly produce many output data, even more than the input. Such operators are classified into the \emph{cache-unfriendly operators} group. The rest are \emph{cache-friendly operators} (Filter, Project, Sort, Limit, etc.). Going back to the example in Section \ref{sec:problem}, because SE2, SE3 and SE4 contains only \emph{cache-friendly operators}, we stop looking at the smaller subexpressions for more sharing possibilities. SE1, however, has one \emph{cache-unfriendly operator} - Join. Then we can either select this candidate (SE1) or the 2 smaller ones (SE2 and SE3). We may not be able to immediately conclude which selection is the best. It depends on multiple criteria (how much we can save in each case, how much data each produces and the memory available, etc.). Potential options are: $\{SE1, SE2, SE3, (SE2, SE3)]\}$.

We use the \emph{operator fingerprints} as a mean to detect the similar subexpressions. Found SEs  represents the potential candidates for building CE in the next phase. We first describe how to compute it and how to produce only the reasonable good SE candidates. The fingerprint $FP(u)$ of an operator $u$ is computed by:
\[FP(u)= h(H(u) | Sort(FP(u.lchild), FP(u.rchild))) (1)\]
where $h$ is a robust hash function (for example SHA256) and 
\[H(u)=
\begin{cases}
 & h(u.label),\ u= \{Filter, Project\}\\ 
 & h(u.label, u.attributes)),\ otherwise
\end{cases} (2)\]

Node $u$ in the operator tree has its label $u.label$ (operator name) and attributes $u.attributes$ (filter predicate, projection columns, joining conditions, etc.). The $Sort$ in (1) ensures the isomorphic property, for example $TableA\ JOIN\ TableB$ and $TableB\ JOIN\ TableA$ are two isomorphic expressions. If $u$ is a binary node, we have $u.lchild$ and $u.rchild$ as the left and right child. If $u$ is a unary node or a leaf node, (1) will become $FP(u)= h(H(u), FP(u.child))$ and $FP(u)= h(H(u))$ respectively. Note that the fingerprint of a node (an operator) is computed after computing fingerprint of its child(s). Two operators (and its descendants) have the same fingerprint then we call them similar subexpressions. They must have the same tree structure and property. Treating the Filter and Project operations differently in (2) allows us to identify SEs having different filtering predicates and projections. In phase 2, they can be transformed into equivalent expressions sharing a common subexpression. For instance $e1 = Filter_{a>10}(x)$ and $e2 = Filter_{b<30}(x)$ are similar subexpressions, then $e1, e2$ can be transformed into $Filter_{a>10}(Filter_{a>10\ \cup \ b < 30}(x))$ and $Filter_{b<30}(Filter_{a>10\ \cup \ b < 30}(x))$ without changing the expression's result.

Now that we have a mean to compare two operator trees, the algorithm \ref{sec:common_sub_alg} searches for SEs among a set of trees while avoid producing worse candidates.

\begin{algorithm}
	\caption{Build hash tree}\label{sec:buildht_alg}
	Input: a logical plans (tree)\\
	Output: HashMap[node, fingerprint]
	\begin{algorithmic}[1]
		\Procedure{$BuildHashTree([T])$}{}
		\State $HT:\{(node, fingerprint)\}=$ empty hash tree
		\For{each node $u$ in $T$}
			\State compute fingerprint $opFT$ for $u$ using (1) and (2)
			\State put $(u, opFT)$ into $HT$
		\EndFor
		\State \Return  $HT$
		\EndProcedure
	\end{algorithmic}
\end{algorithm}

\begin{algorithm}
\caption{Identify similar subexpressions}\label{sec:common_sub_alg}
Input: Array of logical plans (trees)\\
Output: List of SEs put together in group
\begin{algorithmic}[1]
\Procedure{$IdentifySEs([T_{1}, T_{2}, ... T_{N}])$}{}
\State $FT:\{(fp, [nodes])\} =$ empty fingerprint table
%\State $common\_sub\_expressions = \{\}$
\For{each tree $T_{i}, i = 1..N$}
	\State$HT(i) = BuildHashTree(T_{i})$
	\State $AllowedMatching = True$	
	\For{each node $u$ in $T_{i}$ follow the DFS traversal}
		\State $opFP = HT(i)(u)$
		\State $IsMatched = FT$ contains $opFP$
		\If {$AllowedMatching$}
				\State Add $(opFP, u)$ entry to $FT$
		\EndIf
		\State $AllowedMatching = True$	
		\If {$IsMatched$ $AND$ the tree rooted at u does not has cache-unfriendly operator}
			\State Stop the search on u's descendants
		\ElsIf {$IsMatched$ $AND$ the tree rooted at u has cache-unfriendly operator $AND$ $u$ is cache-friendly operator}
			\State $AllowedMatching = False$	
		\EndIf		
	\EndFor
	
\EndFor
\State remove any entries (fp, [nodes]) in FT that has length(nodes) = 1
\State \Return  $FT$
\EndProcedure
\end{algorithmic}
\end{algorithm}

By using the expression (1) and (2), the algorithm \ref{sec:buildht_alg} builds a HashTree for each input tree (line 4) of algorithm \ref{sec:common_sub_alg}. Algorithm \ref{sec:buildht_alg} just shows the simplified pseudo code. (Dynamic programming can be applied to save the computations in the for loop). A single fingerprint table (line 2) is used for tracking the matches. Trees with the same fingerprint (same key) will be put in the same group. Following the depth-first-search, each tree and its \emph{operator fingerprint} are added to the fingerprint table (line 10). As being discussed, if we encounter a match on a tree having only cache-friendly operators, then we can safely skip the search on the descendants (line 13 and 14). A group of SEs is detected whenever that group has 2 or more SEs (line 20). Algorithm \ref{sec:common_sub_alg} has O(N*M) complexity where N is the number of trees and M is the average number of nodes per tree.
%%%%%%%%%%%%%%%%%%%%%%%%%%%%%%%%%%%%%%%%%%%%%%%%%%%%%%%%%%%%%%%%%%%%%%%%%%%%%%%

%%%%%%%%%%%%%%%%%%%%%%%%%%%%%%%%%%%%%%%%%%%%%%%%%%%%%%%%%%%%%%%%%%%%%%%%%%%%%%%
\subsection{Building Covering SubExpression}
\label{sec:covering_subexpression}
We identified the potential sharing candidates (SEs) from the input queries, it is the goal of this phase to build the covering subexpressions (CEs) that are the ``covering" computations. Given the sets of SEs, the optimizer first tries to eliminate obviously bad candidates for sharing with the help of cardinality and cost estimation described in Section \ref{sec:cardinality}. Then for each set of SEs, we construct a single CE to produce all common sharing tuples for the consumers (all SEs in that set). Some CEs might become \emph{cache plans} after the cost-based optimization in phase 3. 

In this paper, we define the \emph{cache plan} as the materialization for our work sharing idea. A \emph{cache plan} is a CE defining the result of a materialized view. In other words, cache plans can be seen as temporary views in RDBMSes and to be materialized in RAM. In our work, \emph{cache plans} (selected CEs) will be executed only once and the result is materialized in RAM so that it could be reused multiple times to achieve work sharing. When a query executes, the engine searches in the distributed memory cache to determine whether the results already exists. If yes, it retrieves the result instead. If no, it executes, return the results as output and store them back to the memory cache.

%%%%%%%%%%%%%%%%%%%%%%%%%%%%%%%%%%%%%%%%%%%%%%%%%%%%%%%%%%%%%%%%%%%%%%%%%%%%%%%
\subsubsection{Pruning bad SEs}
\label{sec:se-prune}
The size of the \emph{cache plans} not only affects the memory occupation but also brings materializing costs

the saving comes from 

As mentioned in Section \ref{sec:problem}, . 


Intuitively, good candidates of \emph{cache plans} are those satisfy the memory constraint while (1) have high frequent of use and/or (2) are expensive to (re)compute, for example producing small amount of output while reading and computing a large amount of input data. Thus, cheap SEs (fast to compute) and heavy SEs (their output exceeds the memory constraint) should be eliminated early from consideration of building the CE. We rather compute them from scratch than just gain a small benefit while paying a big cost in caching a large amount of data. The cost and cardinality estimation we propose in the next section is used to estimate the execution cost and output size of a query.




%%%%%%%%%%%%%%%%%%%%%%%%%%%%%%%%%%%%%%%%%%%%%%%%%%%%%%%%%%%%%%%%%%%%%%%%%%%%%%%

%%%%%%%%%%%%%%%%%%%%%%%%%%%%%%%%%%%%%%%%%%%%%%%%%%%%%%%%%%%%%%%%%%%%%%%%%%%%%%%
\subsubsection{CE construction}
\label{sec:ce-construction}
A CE will be built for each group of SEs. Obviously, for SEs that are actually identical expressions, the CE to be built is exactly the same as the SE. Otherwise, we have to apply some transformations on the SEs. The CE is constructed top-down by OR-ing the filtering predicates and combining the projection columns in the SEs. By doing so, the CE could ``cover'' all records required by its consumers. We also remove duplicated predicates to simplify the operators. Figure \ref{fig:covering} illustrates an example of building the covering subexpression for 2 simple SEs. Traversing top-down 2 trees at the same time, whenever projection or filering operators are encountered, we combine their attributes.

\begin{figure}[!htb]
	\centering
	\includegraphics[scale=0.75]{figures/covering}
	\caption{Building covering subexpression example. The first and second trees are two SEs. The third tree is the CE}
	\label{fig:covering}
\end{figure}
%%%%%%%%%%%%%%%%%%%%%%%%%%%%%%%%%%%%%%%%%%%%%%%%%%%%%%%%%%%%%%%%%%%%%%%%%%%%%%%
%%%%%%%%%%%%%%%%%%%%%%%%%%%%%%%%%%%%%%%%%%%%%%%%%%%%%%%%%%%%%%%%%%%%%%%%%%%%%%%

%%%%%%%%%%%%%%%%%%%%%%%%%%%%%%%%%%%%%%%%%%%%%%%%%%%%%%%%%%%%%%%%%%%%%%%%%%%%%%%
\subsection{Cost-based optimization}
\label{sec:cbo}
%%%%%%%%%%%%%%%%%%%%%%%%%%%%%%%%%%%%%%%%%%%%%%%%%%%%%%%%%%%%%%%%%%%%%%%%%%%%%%%
\subsubsection{Cardinality and cost estimation}
\label{sec:cardinality}
We design a cardinality and cost estimator to estimate the query's output size and execution cost. Just as traditional work, the system analyzes relational operators and uses some pre-computed statistics data of the input tables.

In our work, computing the query's output size mainly relies on the statistics data, which are computed in 2 levels: relation and column. In relation level, the system obtains the number of records and average record size. In more detail level - column, the system collects the min, max, the cardinality and builds an equi-width histogram for each column. The output size of each operator is estimated under the uniform distribution assumption. Although simple, we just want a good estimation to avoid obviously bad plans. More complex histogram techniques could be used to improve the estimation accuracy.

The executing cost comprises of CPU, disk I/O and network costs. Those costs are the results of the multiplication between the pre-defined constants and the estimated number of records.

%%%%%%%%%%%%%%%%%%%%%%%%%%%%%%%%%%%%%%%%%%%%%%%%%%%%%%%%%%%%%%%%%%%%%%%%%%%%%%%

%%%%%%%%%%%%%%%%%%%%%%%%%%%%%%%%%%%%%%%%%%%%%%%%%%%%%%%%%%%%%%%%%%%%%%%%%%%%%%%
\subsubsection{Cache plans selection}
\label{sec:cbo-o}
Our problem is to select the best combination of CEs to form the \emph{cache plans}. For each \emph{CE}, we compute its (profit, weight). Profit of a CE is defined in Section \ref{sec:sharing-vs-notsharing} and weight is the estimated memory footprint of that CE when it is materialized to RAM. Let $W$ be the total cache capacity of the cluster. Then our optimization problem is to maximize the $\sum profit$ under a limited memory capacity: $\sum weight \leq W$. Then our problem is equivalent to the Multiple-choice Knapsack problem (MCKP) and could be solved by many approaches \cite{•}.

Consider $m$ mutually disjoint classes $N_1, N_2, .. N_m$ of items to be packed into a knapsack of capacity $c$. Each item $j \in N_i$ has a profit $p_{ij}$ and a weight $w_{ij}$. The MCKP is to choose at most one item from each class such that the profit sum is maximized without exceeding the capacity $c$ in the corresponding weight sum. The problem is formulated as: 
\begin{align*}
maximize \sum_{i=1}^{k}\sum_{j \in N_i} p_{ij}x_{ij}\\
subject\ to \sum_{i=1}^{k}\sum_{j \in N_i} w_{ij}x_{ij} \leq c\\
\sum_{j \in N_i}x_{ij} \leq 1, i = 1,2,..m\\
x_{ij} \in \{0, 1\}, i = 1,2,..m, j \in N_{i}
\end{align*}

MCKP is a NP-Hard problem. We apply a heuristic algorithm for our system for efficiency \cite{•}.
%%%%%%%%%%%%%%%%%%%%%%%%%%%%%%%%%%%%%%%%%%%%%%%%%%%%%%%%%%%%%%%%%%%%%%%%%%%%%%%

%%%%%%%%%%%%%%%%%%%%%%%%%%%%%%%%%%%%%%%%%%%%%%%%%%%%%%%%%%%%%%%%%%%%%%%%%%%%%%%

%%%%%%%%%%%%%%%%%%%%%%%%%%%%%%%%%%%%%%%%%%%%%%%%%%%%%%%%%%%%%%%%%%%%%%%%%%%%%%%
\subsection{Query Rewriting}
\label{sec:query_rewriting}
Now that we selected the best combination of \emph{cache plans}, the last step is to transform the original input queries to use them. First, the execution engine should be noticed that the output of \emph{cache plans} will be materialized in RAM after its execution. Next, the transformation applied on the input queries is the replacement of SEs by the \emph{extraction plans} on top of the \emph{cache plans}. Remember that \emph{cache plans} are the selected covering subexpressions that covers the records for their consumers. We need to re-apply the filtering and projection to assures the output of queries does not change.

For each input query having the SE and employing the cache plan respectively, we build an \emph{extraction plan} that compensates the cache plan such that the output of the extraction plan on top of the cache plan is equals to the output of that SE. The query rewriting in Figure \ref{fig:rewrite} is a simple example illustrating the technique. In more abstract, we build the \emph{extraction plans} by AND-ing the filtering predicates and combining the top projection columns of the SE.

\begin{figure}[!htb]
	\centering
ubstaint	\includegraphics[scale=0.55]{figures/rewrite}
	\caption{Query rewriting example. The tree in rectangle is the cache plan}
   	\label{fig:rewrite}
\end{figure}
%%%%%%%%%%%%%%%%%%%%%%%%%%%%%%%%%%%%%%%%%%%%%%%%%%%%%%%%%%%%%%%%%%%%%%%%%%%%%%%

We next cover in detail the implementation of our system running on Apache Spark and Spark SQL.